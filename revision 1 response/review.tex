\documentclass{letter}

\usepackage{times,fullpage}

\usepackage{amsmath, amsfonts, amssymb, amscd}
\usepackage{color}
%\usepackage{srcltx}
\usepackage{latexsym,amssymb,amsmath,amsfonts}
\usepackage{verbatim}
\usepackage{bm}
\usepackage{letterbib} 
\usepackage{hyperref}
\usepackage[normalem]{ulem}
\usepackage[shortlabels]{enumitem}

%\newcommand{\gXX}[1]{\color{red} XX #1 XX \color{black}}
%\newcommand{\gsout}[2]{{\sout{#1}}{{\gXX{#2}}}} 
\newcommand{\rr}[1]{{\bf \color{blue}{#1}}}


\newif\ifPDF \ifx\pdfoutput\undefined \PDFfalse \else \ifnum \pdfoutput > 0 \PDFtrue \else\PDFfalse \fi \fi

%\newcommand{\qed}{\nobreak \ifvmode \relax \else
%      \ifdim\lastskip<1.5em \hskip-\lastskip
%      \hskip1.5em plus0em minus0.05em \fi \nobreak
%      \vrule height0.25em width0.5em depth0.25em\fi}

%\usepackage[usenames]{color}
\usepackage[normalem]{ulem}
\newcommand{\gXX}[1]{\color{red} XX #1 XX \color{black}}
\newcommand{\hXX}[1]{\color{green} XX #1 XX \color{black}}
\newcommand{\gcmargin}[2]{{\color{red}#1}\marginpar{\color{red}\tiny\raggedright \bf [GC] #2}}
\newcommand{\hkmargin}[2]{{\color{red}#1}\marginpar{\color{green}\tiny\raggedright \bf [HK] #2}}
\newcommand{\gsec}[1]{\fcolorbox{DarkRed}{DarkRed}{\parbox{3in}{\raggedright \color{green} #1 }}}
\newcommand{\gsout}[2]{{\sout{#1}}{{\gXX{#2}}}}



\newcommand{\bea}{\begin{eqnarray}}
\newcommand{\eea}{\end{eqnarray}}
\newcommand{\be}{\begin{equation}}
\newcommand{\ee}{\end{equation}}
\newcommand{\bi}{\begin{itemize}}
\newcommand{\ei}{\end{itemize}}
%\newcommand{\XX}{{\bf XX}} %\XX
\newcommand{\nn}{{\nonumber}}
\newcommand{\motiv}{{\bf Why, How, Who cares?}}
\newcommand{\mean}{{\bf E}}
\newcommand{\meanw}{{\bf E}_w}
\newcommand{\boldp}{\bf P}
\newcommand{\boldsp}{\bf p}
\newcommand{\cov}{{\bf \Sigma}}
\newcommand{\maxmu}{{\max_\mu}}
\newcommand{\maxtmu}{{\max_{\tilde{\mu}}}}
\newcommand{\tmu}{\tilde{\mu}}
\newcommand{\hmu}{\hat{\mu}}
\newcommand{\mui}{\mu^i}
\newcommand{\mukki}{\mu^i_{k|k}}
\newcommand{\mukkj}{\mu^j_{k|k}}
\newcommand{\mukpki}{\mu^i_{k+1|k}}
\newcommand{\mukpkpi}{\mu^i_{k+1|k+1}}
\newcommand{\x}{\mathcal{X}}
\newcommand{\iso}{\boldsymbol{1_s}}
\newcommand{\dd}{\boldsymbol{1_{di}}}

\newcommand{\tPi}{\tilde{\Pi}}
\newcommand{\tpi}{\tilde{\pi}}
\newcommand{\tLambda}{\tilde{\Lambda}}
\newcommand{\hatxk}{\hat{x}_{k|k}}
\newcommand{\hatxkk}{\hat{x}_{k+1|k+1}}
\newcommand{\hatxkki}{\hat{x}_{k+1|k+1}^i}
\newcommand{\hatxki}{\hat{x}_{k|k}^i}
\newcommand{\Pk}{P_{k|k}}
\newcommand{\Pkk}{P_{k+1|k+1}}
\newcommand{\Pki}{P_{k|k}^i}
\newcommand{\Pkki}{P_{k+1|k+1}^i}

\newcommand{\matlabpathrmmmeas}{C:/MATLAB7/work/UMC/RobustMM/UAV_TrackingMM/MeasurementUpdate}
\newcommand{\matlabpathrmmd}{C:/MATLAB7/work/UMC/RobustMM/UAV_TrackingMM/Version1}
\newcommand{\matlabpathrmmvv}{C:/MATLAB7/work/UMC/RobustMM/UAV_TrackingMM/Version4}
\newcommand{\matlabpathrmm}{C:/MATLAB7/work/UMC/RobustMM/UAV_TrackingMM/Version3}
\newcommand{\matlabpathrmmtwod}{C:/MATLAB7/work/UMC/RobustMM/UAV_TrackingMM/Version3/Figs/}
\newcommand{\matlabpathrmmprob}{C:/MATLAB7/work/UMC/RobustMM/ProblemViz}
\newcommand{\robmmpath}{C:/Documents and Settings/Luca F. Bertuccelli/My Documents/Research/UncertainMarkovChain/RobustMM/}
\usepackage{multirow}
\newtheorem{theorem}{Theorem}%[section]
\newtheorem{definition}{Definition}
\newtheorem{proposition}{Proposition}
\newtheorem{corollary}{Corollary}
\newtheorem{lemma}{Lemma}
\newtheorem{remark}{Remark}

% matrix
\newcommand{\bbm}{\begin{bmatrix}}
\newcommand{\ebm}{\end{bmatrix}}

% itemize
%\newcommand{\bi}{\begin{itemize}}
%\newcommand{\ei}{\end{itemize}}

% enumerate
\newcommand{\ben}{\begin{enumerate}}
\newcommand{\een}{\end{enumerate}}

% equationarray
%\newcommand{\bea}{\begin{eqnarray}}
%\newcommand{\eea}{\end{eqnarray}}

% equation
%\newcommand{\be}{\begin{equation}}
%\newcommand{\ee}{\end{equation}}
\newcommand{\bes}{\begin{eqnarray*}}
\newcommand{\ees}{\end{eqnarray*}}

% indent
\newcommand{\nid}{\noindent}

% figure
\newcommand{\bfig}{\begin{figure}}
\newcommand{\efig}{\end{figure}}

% center
\newcommand{\bc}{\begin{center}}
\newcommand{\ec}{\end{center}}

% bullet
\newcommand{\nb}{\noindent $\bullet$\hspace{1.5mm}}

% ignore
\newcommand{\ignore}[1]{}

% spaces
\newcommand{\hs}[1]{\hspace{#1mm}}
\newcommand{\vs}[1]{\vspace{#1mm}}

% lists / enumerations
\newcommand{\beni}[2]
{\begin{enumerate}\setlength{\itemsep}{#1mm}\setlength{\parskip}{#2mm}}

\newcommand{\bii}[2]
{\begin{itemize}\setlength{\itemsep}{#1mm}\setlength{\parskip}{#2mm}}

% comments












% % % % % % % % % % % % % % % % % % % % % % % % % % % % % % % % % % % % % % % % % % % % % % % % % % % % % % % %
% % % % % % % % % % % % % % % % % % % % % % % % % % % % % % % % % % % % % % % % % % % % % % % % % % % % % % % % % %
\newcommand{\al}{\alpha}
%\newcommand{\be}{\beta}
\newcommand{\e}{\epsilon}
\newcommand{\la}{\lambda}
\newcommand{\s}{\sigma}
\newcommand{\p}{\partial}
\newcommand{\R}{\mathbb{R}}
\newcommand{\E}{\mathbb{E}}
\newcommand{\tr}{\mathrm{tr}}
\newcommand{\LI}{\mathcal{L}}
\newcommand{\Co}{\mathcal{C}}
\newcommand{\C}{\mathbb{C}}
\newcommand{\F}{\mathcal{F}}
\newcommand{\W}{\mathcal{W}}
\newcommand{\K}{\mathbf{K}}
\newcommand{\Z}{\mathbf{Z}}
\newcommand{\w}{\mathbf{w}}
\newcommand{\Be}{\mathbf{\beta}}
\newcommand{\y}{\mathbf{y}}
\DeclareMathOperator{\GP}{GP}
\newcommand{\BV}{\mathcal{BV}}
\newcommand{\Hi}{\mathcal{H}}
\newcommand{\No}{\mathcal{N}}
\newcommand{\V}{\mathbb V}
\newcommand{\La}{\mathcal{L}}


%\def\i{\infty}
\def\l{\langle}
\def\v{\varphi}
\def\r{\rangle}
\def\m{\mathbf}
\def\H{\mathcal{H}}
\def\O{\mathcal{O}}
\def\P{\mathbb{P}}
\def\Ny{ Nystr\"{o}m }

% define paper-specific macros

% Hilbert space/basis function macros
\newcommand{\bfunc}{\phi}
\newcommand{\bvect}{\Phi}
\newcommand{\uncertainty}{\Delta}
\newcommand{\weight}{w}
\newcommand{\estweight}{\widehat{w}}
\newcommand{\idweight}{{W^{*}}}
\newcommand{\fweight}{\mathbf{W^*}}
\newcommand{\fspace}{\mathcal{H}}
\newcommand{\dspace}{\mathcal{H}^*}
\newcommand{\fmap}{\psi}
\newcommand{\fdim}{M}
\newcommand{\fevec}{\mathbf{\xi}}
\newcommand{\fefunc}{\mathbf{\phi}}
\newcommand{\feval}{\mathbf{\lambda}}
\newcommand{\fsol}{\mathbf{\varphi}}
\newcommand{\fvect}{\Psi}
\newcommand{\proj}{\gamma}
\newcommand{\sysop}{\mathcal{A}}
\newcommand{\estsysop}{\widehat{A}}
\newcommand{\controlop}{\mathcal{B}}
\newcommand{\measop}{\mathcal{K}}
\newcommand{\estmeasop}{K}
\newcommand{\sloc}{S}
\newcommand{\centers}{C}
\newcommand{\centerscontrol}{D}
\newcommand{\weights}{{w}}

% dynamical system macros
\newcommand{\state}{x}
\newcommand{\astate}{z}
\newcommand{\aastate}{z}%in time varying this should be \bar {z}
\newcommand{\statesize}{n_s}
\newcommand{\isize}{n}
\newcommand{\sdomain}{D_x}
\newcommand{\tdomain}{D_{\tau}}
\newcommand{\tvarsig}{\tau}
\newcommand{\tsize}{m}
\newcommand{\sdefunc}{F}
\newcommand{\sysfunc}{f}
\newcommand{\control}{\delta}
\newcommand{\cdomain}{D_{\delta}}
\newcommand{\csize}{l}
\newcommand{\pcontrol}{\nu}
\newcommand{\asysfunc}{\hat{f}}
\newcommand{\terror}{e}
\newcommand{\reference}{r}
\newcommand{\rstate}{x_{rm}}
\newcommand{\lyap}{V}
\newcommand{\nneval}{\varsigma}
\newcommand{\adElement}{\nu_{ad}}
\newcommand{\measnoise}{\zeta}
\newcommand{\processnoise}{\eta}


% Gaussian process macros
\newcommand{\kernel}{k}
\newcommand{\kernelvec}{\mathbf{k}}
\newcommand{\estKernel}{\hat{k}}
\newcommand{\evalKernel}{k^*}
\newcommand{\kernelM}{K}
\newcommand{\estKernelM}{\hat{K}}
\newcommand{\meas}{y}
\newcommand{\estmeas}{\widehat{y}}
\newcommand{\imean}{m}
\newcommand{\pVar}{\Sigma}
\newcommand{\Covar}{C}
\newcommand{\estMean}{\hat{m}}
\newcommand{\estPVar}{\hat{\Sigma}}
\newcommand{\switchiMean}{m^{\sigma}}
\newcommand{\switchMean}{\hat{m}^{\sigma}}
\newcommand{\errMean}{\epsilon_m}
\newcommand{\switchErrMean}{\epsilon_m^{\sigma}}
\newcommand{\bstack}{Z}
\newcommand{\zstackT}{Z_\tau}
\newcommand{\Tdisc}{\tau}
\newcommand{\etasup}{\eta_{\mathrm{sup}}}
\newcommand{\stateideal}{\bar{x}}
\newcommand{\nuavail}{\nu_{\mathrm{avail}}}
% miscellaneous macros
\newcommand{\switch}{\sigma}
\newcommand{\Index}{\mathcal{I}}

% integral operator and density macros
\newcommand{\idealP}{\ensuremath{{p}}}
\newcommand{\idealK}{\ensuremath{{\m{K}}}}
\newcommand{\eval}{\la}
\newcommand{\efunc}{\ensuremath{\phi}}
\newcommand{\empEfunc}{\ensuremath{\widehat \phi}}
\newcommand{\empEval}{\ensuremath{\widehat \la}}
\newcommand{\delP}{\ensuremath{{p_{\delta}}}}
\newcommand{\delEmpP}{\ensuremath{\widehat p_{\delta}}}
\newcommand{\empP}{\ensuremath{\widehat p}}
\newcommand{\shP}{\ensuremath{\widetilde p}}
\newcommand{\kMax}{\ensuremath{\kappa}}
\newcommand{\probMeas}{\ensuremath{\mu}}

% matrix and vector macros
\newcommand{\empK}{\ensuremath{K}}
\newcommand{\shK}{\ensuremath{\widetilde K}}
\newcommand{\idealshK}{\ensuremath{\widetilde{\m{K}}}}
\newcommand{\shG}{\ensuremath{\widetilde G}}
\newcommand{\CK}{K^{\shCent}}
\newcommand{\quaK}{\ensuremath{\bar{K}}}
\newcommand{\quaC}{\ensuremath{\oline{\mathcal{C}}}}
\newcommand{\evecs}{\ensuremath{U}}
\newcommand{\evalM}{\ensuremath{\Lambda}}
\newcommand{\extrap}{\ensuremath{E}}
\newcommand{\normChoice}{\xi}

\newcommand{\Dist}{\mathcal{D}}
\newcommand{\DDist}{\mathbb{D}}

% dataset and reduced set macros 
\newcommand{\dom}{\Omega}
\newcommand{\domI}{\Omega_I}
\newcommand{\domO}{\Omega_O}
\newcommand{\sampSet}{\mathcal{X}}
\newcommand{\sampSetOther}{\mathcal{Y}}
\newcommand{\sampSetClass}{\mathcal{D}}
\newcommand{\sampSetLong}{\{x_1, \dots, x_{\nsamp}\}}
\newcommand{\sampSetClassLong}{\{(x_1,y_1), \dots, (x_{\nsamp},y_{\nsamp})\}}
\newcommand{\sampSetShort}{\{\astate_i\}_{1}^{\Tdisc}}
\newcommand{\sampSetClassShort}{\{(x_i,y_i)\}_{1}^{\nsamp}}
\newcommand{\card}[1]{\left|{#1}\right|}
\newcommand{\cent}{c}
\newcommand{\nsamp}{N}
\newcommand{\ncent}{M}
\newcommand{\dtime}{\tau}
\newcommand{\ntrain}{n_{t}}
\newcommand{\ntest}{n_{te}}
\newcommand{\neigs}{r}
\newcommand{\shCent}{\mathcal{C}}
\newcommand{\shCentLong}{\{c_1,\dots,c_{\ncent}\}}
\newcommand{\weightSet}{\W}
\newcommand{\weightSetLong}{\{w_1, \dots, w_{\ncent}\}}
\newcommand{\recSet}{\delta X}
\newcommand{\shSet}{S}
\newcommand{\shrad}{\varepsilon}

\newcommand{\G}{\mathcal{G}}
\newcommand{\X}{\mathcal{X}}

\newcommand{\statesizea}{n_1}
\newcommand{\statesizeb}{n_2}
%\newcommand{\statesize2}{n_2}

\DeclareMathOperator*{\argmin}{arg\,min}
\DeclareMathOperator{\HS}{HS}

\newcommand{\eqlabel}[1]{\label{eq:#1}}
\newcommand{\EQref}[1]{Eq.~(\ref{eq:#1})}
\renewcommand{\eqref}[1]{(\ref{eq:#1})}

\newcommand{\EGP}{\text{EGP}}
\newcommand{\swindex}{j}
\newcommand{\RKHS}{RKHS~}



\newcommand{\sgn}{\mathrm{sgn}}
\newcommand{\refm}{\mathrm{ref}}
\newcommand{\des}{\mathrm{des}}
\newcommand{\Sa}{\mathcal{S}}
\newcommand{\A}{\mathcal{A}}
\newcommand{\T}{\mathcal{T}}



% dynamical system macros

\newcommand{\controlfunc}{b}

\newcommand{\acontrolfunc}{\hat{b}}

% Gaussian process macros





\newcommand{\fow}{\Upsilon}
\newcommand{\FoW}{FOW~}

\DeclareMathOperator{\vol}{vol}

% % hybrid systems
\newcommand{\cont}{x}
\newcommand{\discrete}{x^+}

\DeclareMathOperator{\Dir}{Dir}
\DeclareMathOperator{\BP}{BP}
\DeclareMathOperator{\BeP}{BeP}


%from Hassan


%\newcommand{\DiscP}{\mathbb{D}}
\newcommand{\parms}{\Theta}
\newcommand{\MN}{\mathcal{MN}}
\newcommand{\err}{\mathbf{e}}
\newcommand{\errCov}{\mathbf{\Sigma}}
\newcommand{\zero}{\mathbf{0}}
\newcommand{\pastData}{\mathcal{W}}
\newcommand{\pastDataL}{\bar{\mathcal{W}}}
\newcommand{\pastBoth}{\mathcal{D}}
\newcommand{\pastErr}{\mathbf{E}}
\newcommand{\postWish}{\mathbf{S}_{w\bar{w}}}
\newcommand{\postWishLL}{\bar{\mathbf{S}}_{\bar{w}\bar{w}}}
\newcommand{\postWishLLI}{\bar{\mathbf{S}}^{-\swindex}_{\bar{w}\bar{w}}}
\newcommand{\postWishL}{\mathbf{S}_{\bar{w}\bar{w}}}
\newcommand{\postWishS}{\mathbf{S}_{w|\bar{w}}}



\def\i{\infty}
\def\l{\langle}
\def\v{\varphi}
\def\r{\rangle}
\def\m{\mathbf}
\def\H{\mathcal{H}}
\def\O{\mathcal{O}}
\def\P{\mathbb{P}}
\def\Na{\mathbb{N}}
\def\Ny{ Nystr\"{o}m }

% define paper-specific macros

% Hilbert space/basis function macros

\newcommand{\nsysop}{\mathcal{G}}

% Gaussian process macros

% dataset and reduced set macros 


% E-GP specific macros
\newcommand{\egpFunc}{f}
\newcommand{\egpDyn}{g}

\newcommand{\EGPFspace}{\mathbb{H}}
\newcommand{\EGPSample}{\mathbb{F}}
\newcommand{\EGPFspaces}{\coprod_{j=1}^{\infty}\fspace^j}



% VV Stuff
\newcommand{\Kr}{\m{K}}
\newcommand{\f}{\m{f}}
%\newcommand{\I}{\m{I}}


\DeclareMathOperator{\Var}{Var}
\DeclareMathOperator{\grad}{grad}
\DeclareMathOperator{\diag}{diag}
\DeclareMathOperator{\MMD}{MMD}
\DeclareMathOperator{\divergence}{div}
\DeclareMathOperator{\Span}{span}
%\DeclareMathOperator{\span}{span}
\DeclareMathOperator{\Rank}{rank}
\DeclareMathOperator{\Pro}{\mathbb{P}}
\DeclareMathOperator*{\argmax}{arg\,max}

%\newcommand{\statesize2}{n_2}


\DeclareMathOperator{\DiscP}{Disc}
\DeclareMathOperator{\IW}{IW}
\DeclareMathOperator{\Tr}{Tr}


%\renewcommand{\eqref}[1]{(\ref{eq:#1})}

\newcommand{\NMBRL}{N-MBRL}


% new macros
\newcommand{\dimI}{\ensuremath{D}}
\newcommand{\dimO}{\ensuremath{d}}
\newcommand{\dimP}{\ensuremath{P}}
\newcommand{\sO}{\mathcal{S}}
\newcommand{\risk}{\mathop{\mathbf{R}}}
\newcommand{\empRisk}{\mathop{\widehat{\mathbf{R}}}}
\newcommand{\radCom}{{\mathcal{R}}}
\newcommand{\funcClass}[1]{\mathcal{F}_{#1}}
\newcommand{\empFuncClass}[1]{\widehat{\mathcal{F}}_{#1}}
\newcommand{\oweights}{\ensuremath{\alpha}}
\newcommand{\loss}{\ell}

% newer macros
\newcommand{\Obs}{\mathcal{O}}
\newcommand{\otime}{L}
\newcommand{\JorMul}{O}
\newcommand{\Tset}{\Upsilon}
\newcommand{\measmat}{\Phi}
\newcommand{\enrgy}{\gamma}
\newcommand{\eenrgy}{\lambda}
\newcommand{\blockmat}{\boldsymbol\Phi}
\newcommand{\modker}{\widehat\kernel}
\newcommand{\centker}{\widetilde\kernel}
\newcommand{\Dim}{D}
\newcommand{\Ind}{\mathcal{I}}
\newcommand{\LaVec}{\boldsymbol{\la}}
\newcommand{\LaMat}{\boldsymbol{\Lambda}}
\newcommand{\JorP}{P}
\newcommand{\JorLa}{\Lambda}
\newcommand{\empKDiag}{\mathbf{K}^D}
%\newcommand{\empKMod}{\widetilde{\empK}} %repeated twice
\newcommand{\empKBlock}{\widehat{\mathbf{K}}} % now catempK
%\newcommand{\ABlock}{\widehat{\mathbf{A}}} %now defined as \catdualopC
\newcommand{\Coh}{\mathcal{C}}

% feature map macros
\newcommand{\fspaceC}{\ensuremath{\mathcal{H}^{\mathcal{C}}}}
\newcommand{\fspaceD}{\ensuremath{\widetilde{\fspace}}}
\newcommand{\functionSet}{\mathcal{J}}
\newcommand{\AtomsControl}{\mathcal{F}_{\centerscontrol}}
\newcommand{\fspaceApprox}{\widehat{\fspace}}
\newcommand{\fsubspaceApprox}[1]{\fspaceApprox_{#1}}
\newcommand{\fmapApprox}{\widehat{\fmap}}
\newcommand{\fbasis}{\psi}
\newcommand{\fspaceEl}{f}
\newcommand{\fspaceApproxEl}{\widehat{\fspaceEl}}
\newcommand{\obsMat}{\empK}
\newcommand{\randFreq}{\omega}
\newcommand{\randMat}{V}
\newcommand{\obsMatRow}{\widehat{\Psi}}
\newcommand{\Kiprod}[2]{\langle \fmap(#1),\fmap(#2) \rangle_{\fspace}}
\newcommand{\Atoms}{\mathcal{F}^{\shCent}}

% controls specific stuff
\newcommand{\weightmap}{\ensuremath{\mathcal{W}}}
\newcommand{\weightmapI}{\ensuremath{\mathcal{W}^{-1}}}
\newcommand{\weightmapC}{\ensuremath{\acute{\mathcal{W}}}}
\newcommand{\weightmapCI}{\ensuremath{\acute{\mathcal{W}}^{-1}}}
\newcommand{\data}{X}
\newcommand{\ncontrol}{\ell'}
\newcommand{\weightc}{\acute{\weight}}
\newcommand{\empKD}{\empK_{\centerscontrol}}
\newcommand{\empKCD}{\empK_{\centers\centerscontrol}}
\newcommand{\ControlMat}{\Psi}
\newcommand{\controlCent}{\mathcal{D}}
\newcommand{\controlCentLong}{\{d_1,\dots,d_{\ncontrol}\}}
\newcommand{\cbasis}{\widetilde{\psi}}

% minimal polynomial and measurement map etc
\newcommand{\charpoly}{\pi}
\newcommand{\minpoly}{\alpha}
\newcommand{\minpolyv}{\xi}
\newcommand{\geomMult}[1]{\gamma_{#1}}
\newcommand{\algMult}[1]{\mu_{#1}}
\newcommand{\genMult}[1]{\nu_{#1}}
\newcommand{\rank}{r}
\newcommand{\nevals}{r}
\newcommand{\linvec}{v}
\newcommand{\polycoeff}{a}
\newcommand{\acycdeg}{m}
\newcommand{\minmeas}{\ell}
\newcommand{\linspace}{\mathcal{V}}
\newcommand{\linspaceout}{\mathcal{U}}
\newcommand{\subspace}{\linspace_{\mathcal{S}}}
\newcommand{\measmap}{\widetilde{\empK}}
\newcommand{\premeasmap}{\mathring{\empK}}
\newcommand{\measmapsum}{\check{\empK}}
\newcommand{\mmapInd}[1]{\mathcal{J}^{({#1})}}
\newcommand{\restrict}[2]{{#1}_{|#2}}
\newcommand{\fsubspaceC}[1]{\fspaceApprox_{#1}}
\newcommand{\initcondf}{\widetilde{f}}
\newcommand{\initcond}{\weight_0}
\newcommand{\sysopC}{\sysop_{\shCent}}
\newcommand{\sysopApprox}{\widehat{\sysop}}
\newcommand{\dualop}{A}
\newcommand{\dualopC}{A_{\shCent}}
\newcommand{\dualopApprox}{\widehat{\dualop}}
\newcommand{\empKShadFull}{\mathbf{\empK}}

\newcommand{\sampind}{i}
\newcommand{\centind}{j}
\newcommand{\setind}{\iota}

% kernel observer transition macros
\newcommand{\tindex}{\tau}
\newcommand{\ftime}{T}
\newcommand{\orbit}{\boldsymbol{\Theta}}
\newcommand{\banachspace}{\mathcal{B}}
\newcommand{\banachop}{\mathcal{L}}
\newcommand{\banachfunc}{u}
\newcommand{\rands}{\varsigma}

\DeclareMathOperator{\dims}{dim}
\DeclareMathOperator{\degs}{deg}
\DeclareMathOperator{\Ker}{Ker}

%Borrowed from CDC
\newcommand{\evalvec}{\boldsymbol{\eval}}
% supplementary macros
\newcommand{\transempK}{\breve{\empK}}
\newcommand{\catempK}{\widehat{\mathbf{K}}}
\newcommand{\catdualopC}{\widehat{\mathbf{A}}}
\newcommand{\catJorP}{\boldsymbol{\JorP}}
\newcommand{\catJorLa}{\boldsymbol{\JorLa}}
\newcommand{\estdualopC}{\widehat{A}_{\shCent}}
\newcommand{\estcontrolop}{\widehat{B}}

% decomposition macros
\newcommand{\Fmap}{\Psi}
%\newcommand{\LaMat}{\boldsymbol{\Lambda}}
%\newcommand{\JorP}{P}
%\newcommand{\JorLa}{\Lambda}
\newcommand{\FrobP}{Q}
\newcommand{\FrobC}{C}
\newcommand{\subsetC}[1]{\mathcal{S}^{(#1)}}
%\newcommand{\Atoms}{\Fmap}
\newcommand{\atomSubset}[1]{\Fmap^{(#1)}}
\newcommand{\centerSubset}[1]{\shCent^{(#1)}}

% more macros for proofs
\newcommand{\zerosMat}[2]{\mathbf{0}_{\R^{#1\times#2}}}
\newcommand{\zerosK}{\zerosMat{\nsamp}{\ncent}}
\newcommand{\zerosU}{\zerosMat{\nsamp}{\nsamp}}
\newcommand{\bdU}{\mathcal{U}}

% sensor placement macro
\newcommand{\entireSet}{\mathcal{V}}
\newcommand{\xset}{X}
\newcommand{\sensub}{\mathcal{A}}
\newcommand{\sensubh}{\bar{\sensub}}
\newcommand{\xdomvec}{\bf x}
\newcommand{\ydomvec}{\bf y}
\newcommand{\wvec}{\bm w}
\newcommand{\fest}{\hat{f}_\tindex}
\newcommand{\measure}{\nu}
\newcommand{\meanDist}{\mu}


\setlength{\topmargin}{0in}
\setlength{\textheight}{9in}
\setlength{\textwidth}{6.5in}
\setlength{\oddsidemargin}{0in}
\setlength{\evensidemargin}{0in}

\begin{document}

\section{Statement of Revision}

We thank the associate editor and reviewers for their useful comments and suggestion. We have significantly revised the manuscript to reflect the reviewer comments and concerns. Wherever possible, we have marked the changes within our manuscript in \rr{blue}.

In the following, we provide an account of the changes that have been made.  The nature of
the review for this paper is such that the comments were provided both
in paragraph form and in bulleted form.  When addressing the paragraph
form, we have identify the comment using the notation {\bf {\color{red}(Rx:Py)}} where 
{\bf x} is the Reviewer number and {\bf y} is paragraph referred to.
When addressing the bullet form, we utilize the form {\bf (Rx:z)} where
{\bf x} identifies the Reviewer and {\bf z} identifies the bulleted comment.
Each bulleted comment included in the listing is followed by a description
of modifications to the paper addressing the comment. Furthermore, changes in the paper have been marked with a color coding.  We hope that this
method is sufficiently clear.
We would like to thank the reviewers for their time and effort at providing
feedback in order to improve the exposition of the paper.

\noindent\textbf{An overview of the major changes}

In response to reviewer comments, we have undertaken some major revisions for this paper. Here we summarize some of those major revisions, further details are provided in response to individual reviewers:

\begin{enumerate}
\item We have included proofs of all main theorems in the appendix.
\item We have included new sections on generating path for a single mobile agents to observe a dynamically varying phenomena.
\item We have added theoretical results into links between our model and Koopman operator theory
\item We have shown how those results can be used for the problem of path planning for mobile sensing agents seeking to estimate the state of the system
\item We have significantly re-focused the control problems in agriculture sidebar.
\item We have included several references in the related works section to better outline the contributions of this work in the context of the vast literature out there on Gaussian Processes, spatiotemporal modeling, and distributed sensing.
\end{enumerate}

\section{Response to comments by Associate Editor}

In this paper the authors provide an overview of methods and algorithms
to build data-driven models that may be used for estimation and
prediction of spatio-temporal processes. 

Four reviews were obtained for this paper. In general, the reviewers
agree that the topic of this paper is of interest to the control
community, and that the proposed approach appears sound. However, the
reviewers point out a number of serious concerns, in particular:

Comparison with the state of the art: the comparison with the state
of the art appears insufficient (see Reviewers 5, 6, and 7). This makes
the contribution of this paper unclear (see Reviewer 6).

Presentation style and rigor: the presentation style is quite poor
(see Reviewers 5, 6, 7, and 9). In particular, the paper contains
several typos and the presentation of the results is rather dense. Most
importantly, the authors refer to an Appendix for the presentation of
the proofs of the theoretical results; however, such an Appendix is
missing. Collectively, these shortcomings make this paper very
difficult to read and evaluate -- see, in particular, Reviewer 7.
Furthermore, note that CSM papers do not have appendices, 
so the location of these proofs will have to be changed.

Tutorial contribution: the insufficient comparison with the state of
the art, the poor presentation style, and the lack of proofs for the
theoretical results make the tutorial value of this paper quite weak.
In other words, it would be very difficult for a practitioner to use
this paper as a guide for developing estimation and prediction
algorithms for spatio-temporal processes (when should one use the
methods presented in this paper? are there implementation guidelines?
etc.). The authors do reference a software library - in my opinion,
this paper should more explicitly reference such a library in order to
better explain the practical and implementation aspects of the proposed
methodology.

 I concur with the reviewers’ comments and I recommend that the authors
prepare a significantly revised version addressing all the
concerns and suggestions, with a key focus on strengthening the
tutorial value of the paper. In addition, the authors should address
the following comments:

{\color{red}(AE:1)} We thank the associate editor for the opportunity to submit a revision.  We have prepared a significant revision that includes the complete proofs of all of the results presented in this paper. 


The sidebar "Key control problems in agriculture," while
interesting, appears excessively long. The authors should shorten it,
and make its relevance to the topic of this paper clearer.

{\color{red}(AE:2)} We have reduced the key control problems sidebar. We have significantly reduced the length of the sidebar by reducing the general discussion about agriculture. We have focused instead the sidebar on the challenges in spatiotemporal estimation for advancing agricultural robotics.


The quality of many of the figures is quite low (see, for example,
Figure 1, Figure 8, and Figure S8). The authors should make all figures
clearly readable.

{\color{red}(AE:3)} We have bettered the quality of the figures throughout the paper.


\section{Response to comments by Referee 1. reviewer ID 4665}

{\color{red}(R1:P1)} This paper describes an approach, developed by the authors over the
previous few years, to build data-driven models that may be used for
estimation and prediction within spatiotemporal monitoring. The model
is based on predicting a process that evolves in an RKHS, where finite
dimensional measurements are available. Given these measurements and a
choice of kernel, approximation methods for the feature space may be
developed (via e.g. a dictionary of atoms, random fourier features,
etc). Then, given the measurements, a linear dynamical system may be
constructed in the feature space. This finite dimensional feature
evolution may be analyzed from the point of view of linaer systems
theory, and theoretical results on observability and estimation are
stated. 

{\color{red} Response:} The authors thank the anonymous reviewer for his succinct and accurate evaluation of this paper.


{\color{red}(R1:P2)} While I believe this paper is coherent and reasonably strong, there are
a few ways in which it could be improved. 

First, the technique of finding an alternate space in which the state
evolves linearly is not novel (nor is it claimed to be). The paper
would benefit from an extended discussion and comparison to alternative
methods. In particular there has recently been considerable interest in
methods inspired by Koopman analysis, which the authors mention only
extrememly briefly, in passing. Since the authors present experiments
on fluid flow problems, they should present comparisons to dynamic mode
decomposition, which has been a popular technique and has many
similarities to the method presented in this work. See, e.g.,
Schmid, Peter J. "Dynamic mode decomposition of numerical and
experimental data." Journal of fluid mechanics 656 (2010): 5-28.
and the many articles that cite the above. Moreover, there is a broad
literature extending dynamic mode decomposition and other
Koopman-inspired techniques. In particular, one may refer to the work
of Steve Brunton and Nathan Kutz, such as 
Brunton, Steven L., Joshua L. Proctor, and J. Nathan Kutz. "Discovering
governing equations from data by sparse identification of nonlinear
dynamical systems." Proceedings of the National Academy of Sciences
113.15 (2016): 3932-3937.

While these works do not explicitly discuss the estimation problem,
comparisons between the chosen model and those discussed above would
greatly improve the paper and allow readers to better appreciate and
situate the proposed methods within the literature. 

{\color{red}(Response)} We have included in this revision an extensive analysis of this our methods in terms Koopman operator theory, including proofs of key theoretical results. To demonstrate those results, we have included a comparison of the modes derived from our model to those obtained via dynamic mode decomposition (DMD). We have also included recent research into how the modes can be utilized to generate best paths for moving agents seeking to infer the current state of a system. This amounts to around six new pages of material, primarily in pages 20-25.

{\color{red}(R1:P3)} In addition the above, I was confused by the presentation of the
theoretical results. I could not find the appendix in which proof of
the theoretical results was presented, and so have not been able to
review them. 

{\color{red}(Response)} You should be able to find the appendix with all the theoretical proofs on page 33 of the revised paper.

{\color{red}(R1:P4)} The evolving GP formulation, in which dynamics are shared across
dynamical systems should not behave similarly, was confusing. In
particular, the definition of similar fluid systems seems ad hoc, and
this section must be better motivated and explained in general. 

{\color{red}(Response)}


{\color{red}(R1:P5)} Minor comments
---------------
\begin{enumerate}
\item page 1 line 10, "spatiotemporally" misspelled
\item page 4 line 24 "interpretability" misspelled
\item text very small in figure 8, figure 10, figure S8. Generally, the
text size of the figures needs to be fixed throughout.
\end{enumerate}

{\color{red}(Response)} Thank you for pointing out these typos; they have been fixed.



\section{Response to comments by Referee 6. reviewer ID 4667}

{\color{red}(R6:P1)} In this paper, the authors study how to perform state estimation of
time-varying spatio-temporal processes, and how to design sensors
placement. 
The topic is of interest for the community, however, the
paper needs significant changes and work, both on the scientific side
and writing style. 

{\color{red}(Response)} We thank the reviewer for their interest. We have significantly revised the paper to address the concerns raised by the reviewers.

{\color{red}(R6:P2)} The authors claim that one of the two main contributions is 

"we demonstrate that spatiotemporal functional evolution can be modeled
using stationary kernels with a linear dynamical systems layer on their
mixing weights",

however, to me, this is not a novelty and there is a huge literature
about the estimation of time-varying spatio-temporal processes and I
would recommend the authors to compare their work with the following
papers, highlighting the differences
\begin{enumerate}
\item J. Hartikainen and S. Sarkka. Kalman filtering and smoothing
solutions to temporal Gaussian process regression models. In Machine
Learning for Signal Processing (MLSP), 2010 IEEE International
Workshop on, pages 379ñ384. IEEE, 2010.
\item J. Hartikainen, J. Riihimaki, and S. Sarkka. Sparse spatio-temporal
gaussian processes with general likelihoods. In Artificial Neural
Networks and Machine LearningñICANN 2011, pages 193ñ200.
Springer, 2011.
\item A. Carron, M. Todescato, R. Carli, L. Schenato, and G. Pillonetto.
Machine learning meets Kalman filtering. In 55th Conference on
Decision and Control. IEEE, December 2016.
\item J. Hartikainen. Sequential Inference for Latent Temporal Gaussian
Process Models. PhD thesis, Aalto University, 2013.
\item S. Sarkka and R. Piche. On convergence and accuracy of statespace
approximations of squared exponential covariance functions.
In Machine Learning for Signal Processing (MLSP), 2014 IEEE
International Workshop on, pages 1ñ6. IEEE, 2014.
\item S. Sarkka, A. Solin, and J. Hartikainen. Spatiotemporal learning
via infinite-dimensional bayesian filtering and smoothing: A look
at Gaussian process regression through kalman filtering. Signal
Processing Magazine, IEEE, 30(4):51ñ61, 2013.
\item TT Ho et.al., Multiresolution stochastic models for
the efficient solution of large-scale space time estimation
problems, Proc. IEEE ICASSP, 1996
\item Noh J, Solo V, Testing for Space-time Separability in
functional MRI. Proc IEEE Int Symposium on Biomedical
Imaging, pp 412-415,Washington DC, USA, April
2007.
\item J. Noh and V. Solo, Space-Time Separability in FMRI:
Asymptotic Power Analysis and Cramer- Rao Lower
Bounds, IEEE Trans. Sig. Proc., (2013), 61(1), pp
148-153.
\item JR Stroud et al, Dynamic models for spatio-temporal
data, JRSSB,63, pp673-689, 2001.
\item RN Miller, Toward the application of the Kalman filter
to regional open ocean modelling, Jl Phys Oceanog.,
16, 72-86, 1986.
\item F Lindgren and Havard Rue and J Lindstrom, An explicit
link between Gaussian fields and Gaussian Markov
random fields: The SPDE approach, Jl Royal Stat.
Soc. B, 2011.
\end{enumerate}


{\color{red}(Response)}: The reviewer is indeed correct, there is a very large body of work on modeling spatiotemporally varying dynamic systems in the literature. Our work is inspired from this literature, and our main contribution is in realizing that the Hilbert spaces produced by Kernel methods can embed dynamical systems, and then utilizing those dynamical systems for doing estimation and control. The prior literature  spans from that of statistical modeling to modern machine learning methods such as LSTMs. In particular, there is also a very large body of work on kernel methods, and Gaussian processes to modeling spatiotemporally varying functions. A large number of those papers have been cited in the related work. We have now made sure that all of the papers suggested by reviewer have also been cited. With respect to those papers, and many similar ones, what is new with E-GP is its ability to learn end-to-end complex spatiotemporal dynamic patterns such as solutions to Navier Stokes equations, and its ability to approximate the eigenmodes of Koopman decomposition (please see response to reviewer 1). 

We have revised the paper to address the reviewer concern in the following manner:

We have added the below statement at the beginning of related work section:

\rr{There is a very large amount of literature on Gaussian Processes and spatiotemporal modeling, a complete survey of this literature is beyond the scope of this paper. Since our contributions are in the area of creating a feedback based observer in the feature spaces of GP models, we discuss here related work in three related areas: Spatiotemporal modeling with GPs, Kalman filtering and GP connection, and sensor placement for inference in spatiotemporal domains.}

We then proceed to discuss the various modeling methods mentioned by the reviewer:

\rr{Other approaches that utilize hierarchy or evolution of kernels have also been used for modeling spatiotemporal functions \cite{hartikainen2013sequential,lindgren2011explicit,ho1996multiresolution}}.

\rr{A more clever approach is to use state-space representations of time-varying GPs \cite{sarkka2014convergence,hartikainen2013sequential}. In this view point, each GP instance is viewed as a snapshot of an evolving set of weights. We follow in a similar vein here, with added emphasis on understanding the mathematical structures leading to observability and controllability.}

\rr{This approach has been utilized previously in MRI imaging \cite{noh2007testing,noh2012space}}.

\rr{Kalman filtering in the context of Gaussian processes and Kernel models has also been quite widely studied \cite{carron2016machine,hartikainen2010kalman,sarkka2013spatiotemporal,stroud2001dynamic,miller1986toward}.  There is clearly a direct link between the Bayesian approach to inference taken in GPs and its natural extension to Kalman Filters. Our contributions here are in creating explicit connections between feedback observers and inference by deriving conditions of observability in the kernel space. This leads to explicit conditions on the number of sensors required and where to place them.}


%\rr{As can be seen, the kernel based approach to modeling spatiotemporally varying system has received significant attention in the literature. Our work here is indeed inspired by this rich body of literature. The significance of our main contribution is in demonstrating how dynamical systems can be embedded within the Reproducing Kernel Hilbert Spaces of kernel models and utilizing those embeddings for estimation and control. From that perspective, embedding a linear dynamical system in the RKHS is the simplest and the most straight-forward part of the contribution, and has indeed been tried before. The significance of our contribution is in showcasing the elegance of this approach is by providing theoretical insights to discovering hidden spatiotemporal structures in the RKHS and using those for estimation and control. We strongly believe that our results indicate that such embeddings in feature spaces of machine learning models are indeed going to be key to bring machine learning to solve meaningful control problems.}
specific comparisons, discuss the work on reservoir computing, cite Brockett

{\color{red}(R6:P2)} Also, the second contribution is not clear to me, what are the
differences with the previous work [19], [20],[21]? 

{\color{red}(Response)} This paper represents a single unifying body of work bringing together out previous work and linking it together. In addition, there are significant new contributions introduced in this paper, including results relating to random sensor placement, connection with Koopman operators, and sensor placement with mobile sensors. To clarify the relationship with our prior work, we have added a new section titled: Outline and relationship with author's prior work. In that section, the following statement is added:

\rr{Elements of the work presented in this paper first appeared in Neural Information Processing Systems (NIPS 2016) (\cite{Kingravi16_NIPS,whitman2016NIPSworkshop}), IEEE CDC 2015 conference \cite{Kingravi:2015a}, the Conference on Robot Learning (CoRL 2017) \cite{whitman2017learning} and IEEE ACC 2018 conference \cite{Maske18_ACC}.  This paper presents a comprehensive set of results and fills in the missing links in a single encompassing publication, and introduces new results on observability in the presence of random sensor placement, connection to Koopman operators, and generating paths for single observing agents. As such, we have focused in this article mostly on the fundamental theory and practical algorithms for modeling, estimation, and control, while the excruciating details of how to optimally implement the presented algorithms are omitted. Instead an open-source code-base is made available in MATLAB at \url{http://daslab.illinois.edu/software.html} or at \url{https://github.com/hkingravi/FunctionObservers} and in Python on GitHub at \url{https://github.com/hkingravi/funcobspy}. }


{\color{red}(R6:P2)}Additionally, I think that the paper has not been carefully proofread
since it is plenty of mistakes, repetitions, and sentence that are not
appropriate for a journal publication. For example:
\begin{enumerate}
\item The comma after the last name in the authors' list
\item first page: spatiotermpoarlly -> spatiotemporally 
\item first page: this very challenging problem -> this challenging
problem
\item first page: There are many parallel (I would remove parallel)
examples in other fields -> can you provide references?
\item first page: computational packages?
\item second page: has traditionally been in the province? 
\item second page: very high amount of variability -> high variability 
\item second page, line 13-24: can the literature review be structured? 
\item second page: (for the matter) -> remove
\item second page, line 32: can you provide a reference about deep
learning models?
\item fourth page, line 5: machine learning machine learning 
\item fourth page, line 5: fused? 
\item fourth page, line 25: This is mostly because $\to$ This is one of the
reasons 
\item fourth page, line 27: I do not think this is one of the major
challenges 
\item fifth page, line 2-6: here the text is too informal 
\item page 7, line 10: time-varying distribution?
\item page 7, line 17: taxing? 
\item page 7, line 17: realizable? $\to$ feasible 
\end{enumerate}


The list is however way longer. I recommend the authors to clarify
their contributions, extend the literature review, and carefully
proofread the paper. 

{\color{red}(Response)} Thank you for these comments. We have proofread the paper more carefully, and have addressed these issues. 

\section{Response to comments by Referee 7. reviewer ID 4669}

{\color{red}(R7:P1)} My main concern is that there are no proofs of the formal
statements. The authors do say that the proofs are included in the
appendix, but I could not find any appendix, nor any pointer to any
online technical report for them. Therefore, I cannot recommend this
paper for publication until I get a chance to review the proofs.

{\color{red}(Response)} Appendix is now included

{\color{red}(R7:P2)} For example, the notion of shadedness seems to too strong, i.e., it appears
that the sensor matrix needs to ensure that it "sees" all states
directly! On the other hand, the classical notion of observability
leverages the dynamics matrix A to provide a crisp, necessary and
sufficient rank condition. 

{\color{red}(Response)} The definition of shadedness is a simple way of defining coverage of the basis with respect to the sensors in a geometric way with respect the kernel function used. At first sight, it may seem to be too strong; however, if you look at the statement of Proposition 2, it becomes clear that even shadedness is not enough if there is even a single repeated eigenvalue (see our response to {\color{red}(R7:P3)}). In that case, either a measurement map needs to be constructed (Proposition 3) or an $\ell$-shaded matrix is required (Theorem 1). 


{\color{red}(R7:P3)} Proposition 2 is only a sufficient condition. I think the discussion
just prior to Proposition 2 will benefit with the example of a K which
is shaded and yet the system is not observable. 

{\color{red}(Response)} Please see the proof of Proposition 2, which gives such an example for the Gaussian kernel. 


{\color{red}(R7:P4)} The second on random sampling is very vaguely written. There is no
clear ``algorithm" that summarizes the method. Further, how is the
measure $ \nu $ in equation 10 defined? What if the $ p_\epsilon $ = 0? Can
that happen in the present set-up? Theorem 2 talks about an expected
number of randomly placed sensors -- what is this expectation over? There is a $ \delta $ in Theorem 3, but that never appears anywhere in the result. 

{\color{red}(Response)}  Algorithm 2 in the appendix summarizes the method. For two-dimensional domain, the measure $ \nu $ would correspond to area, and for other high dimensional domains it would correspond to volume. Expectation is over the number of sensors (or samples). If the probability $ p_\epsilon $ is 0, the solution would degenerate to infinite number of samples. Definition of $ \delta $ is provided  within Theorem 3's proof given in the appendix.  

{\color{red}(R7:P5)} Further, if one plugs in the value of N equal to the lower bound in Theorem 3, then the probability of unobservability appears to be a number that can be larger than 1 and therefore, the utility of this result appears to be unclear!

{\color{red}(Response)} Thanks for pointing out this flaw. The condition on $ N $ in theorem 3 is now changed to $ N > 2\varsigma/p_\epsilon $, with the proof remaining unaffected. 

\section{Response to comments by Referee 9. reviewer ID 4673}

{\color{red}(R9:P1)} In this article, the authors study the problem of modeling and monitoring of stochastic phenomena with spatiotemporal evolution using the measurements from only a few sensors. The key idea is to bridge system theory and machine learning. To model the evolution of the stochastic phenomenon, they utilize a (linear) time-varying stochastic process, whose mean $f_{\tau}$ evolves temporally and is restricted to lie in a RKHS, which is generated by features used in Gaussian Process modeling. They further approximate the original evolution in functional space using a finite-dimensional evolution.  Controllability and observability of the system (i.e., the process) are achieved by designing a kernel controller and a kernel observer, respectively.

{\color{red}(Response)} The authors thank the anonymous reviewer for his succinct and accurate evaluation of this paper.

{\color{red}(R9:P2)} The authors then provide theoretical results in terms of (sufficient) conditions for observability of the system. A random sensor placement algorithm is proposed. They show an upper bound on the expected number of randomly placed sensors to achieve observability of the system. Furthermore, given a fixed number of randomly placed sensors, the probability that the system is not observable is upper bounded. These theoretical results are then followed by some simulations. However, as pointed out by the authors, the theoretical results in terms of the random sensor placement algorithm are for the case when the system dynamics have a (block) diagonal structure.

{\color{red}(Response)} The reviewer will be interested to see that we have included some new results in this paper dealing with spectral analysis of transition matrices whose structure is not as neatly (block) diagonal.

{\color{red}(R9:P3)} Overall, this article is well written for people who seek to gain some overview of the results in such as research fields. The content and results in this article are abstract, but the authors have tried to make them direct and easy to understand to a broader class of readers. In general, it is the authors’ choice about how much technical stuff should be put into this article. From my point of view, I did find the math not easy to follow when the authors try to put many results into this article. However, they do a good job to explain the intuition behind the theoretical results in the “Discussion of Theoretical Results” section. I believe more of such explanations could make the presentation of this article better. Moreover, as machine learning is supposed to be a keyword of this article, more explanations about how the authors apply the techniques from machine learning to system analysis would also be beneficial. These explanations could potentially use some plain expressions without going to the math.

{\color{red}(Response)} The authors thank the reviewer for his grace in reading and reviewing the materials we have submitted. The reviewer should note that the kernel methods which form the backbone of our Kernel Observer and Evolving Gaussian Processes models are classic machine learning methods.

Here are some detailed comments:
\begin{enumerate}[(1)]
	\item In line 30 on page 3, “provide” should be “provides”.
	\item In line 5 on page 4, ‘’machine learning” is duplicated.
	\item In line 11 on page 6, “dimensions” should be “dimensions”.
	\item In Eq. (6), should the first $\omega_{\tau}$ be $\omega_{\tau+1}$.
	\item Is there an appendix to this article as mentioned in the first line on page 12?
	\item A note on the discussion that the authors make in the first paragraph on page 14. If we already need $M$ sensors to obtain $\hat{A}$ through the system identification phase, does the problem of choosing $N\ll M$ sensors for the estimation phase still make sense?
\end{enumerate}

{\color{red}(Response)} The typos have been corrected. The reviewer may find the appendix located on page 33. In response to (6), the choice to reduce the number of sensors in the estimation phase relative to the system identification phase, in our opinion, makes sense in a lot of situations where there are ongoing costs of operating sensors. Furthermore, based on the results in the sidebar on generalizing across similar systems, it would be possible to (after completing the system identification phase) assign the sensors to estimate the state of several different systems.

\section{Conclusion}

Again, we thank the anonymous reviewers for their comments. We hope we 
addressed all concerns and improved the overall readability of the paper. We are happy to provide further clarification or revisions as requested.





\small
%\bibliographystyle{unsrt}
\bibliographystyle{plain}
\bibliography{../main/BIB/daslab_all,../main/BIB/daslab_pubs,../main/BIB/ACL_all,../main/BIB/ACL_Publications,../main/BIB/bibtex_database_chowdhary_machine_learning,../main/BIB/cybersees,../main/BIB/Robotics}

\end{document}
